\newcommand{\smallbullet}{{\scalebox{0.5}{$\bullet$}}}

\renewcommand{\bar}[1]{\overline{#1}}
\newcommand{\quotient}[2]{#1/\, #2}

\newcommand{\F}{{\mathbb{F}}}
\newcommand{\Z}{{\mathbb{Z}}}
\newcommand{\A}{{\mathbb{A}}}

\newcommand{\M}[2]{\mathsf{M}_{#1}(#2)}

\DeclareMathOperator\End{End}
\DeclareMathOperator\Hom{Hom}
%\DeclareMathOperator\ker{Ker}
\DeclareMathOperator\im{Im}
\DeclareMathOperator\Span{Span}
\DeclareMathOperator\Map{Map}
\DeclareMathOperator\rank{rank}
\DeclareMathOperator\coker{Coker}
\DeclareMathOperator\coim{Coim}
\DeclareMathOperator\nullity{Nullity}
\DeclareMathOperator\id{id}
\DeclareMathOperator\nul{Nul}
\DeclareMathOperator\col{Col}
\DeclareMathOperator\Mor{Mor}
\DeclareMathOperator\Ob{Ob}
\DeclareMathOperator\chart{char}
%\DeclareMathOperator\deg{deg}
%\DeclareMathOperator\dim{dim}
%\DeclareMathOperator\det{det}
\DeclareMathOperator\Aut{Aut}
\DeclareMathOperator\sgn{sgn}
\DeclareMathOperator\adj{adj}
\DeclareMathOperator\tr{tr}
\DeclareMathOperator\proj{proj}

\newcommand{\C}{{\mathcal{C}}}
\newcommand{\D}{{\mathcal{D}}}

\newcommand{\Set}{{\mathbf{Set}}}
\newcommand{\Vect}{{\mathbf{Vec}}}
\newcommand{\Alg}{{\mathbf{Alg}}}
\newcommand{\CAlg}{{\mathbf{CAlg}}}
\newcommand{\SAlg}{{\mathbf{SAlg}}}

\newcommand{\B}{{\mathcal{B}}}
\newcommand{\R}{{\mathbb{R}}}
\newcommand{\T}{{\mathcal{T}^\smallbullet}}

\newcommand{\BL}{{\mathsf{BL}}}

\newcommand{\Sym}{{\mathcal{S}^\smallbullet}}
\newcommand{\Ext}{{\bigwedge^\smallbullet}}
\newcommand{\Env}{{\mathcal{U}}}
\newcommand{\ideal}{{\mathcal{I}}}

\newcommand{\GL}{{\mathsf{GL}}}
\newcommand{\Orth}{{\mathsf{O}}}
\newcommand{\SO}{{\mathsf{SO}}}
\newcommand{\Uni}{{\mathsf{U}}}
\newcommand{\SU}{{\mathsf{SU}}}
\newcommand{\SL}{{\mathsf{SL}}}

% For arrows pointing
\tikzset{%
	myarrow/.style = {-Stealth, shorten >=5pt}
}
\newcommand{\mypoint}[2]{\tikz[remember picture]{\node[inner sep=0, anchor=base](#1){$#2$};}}

% For arrows inside paths
\tikzset{
	set arrow inside/.code={\pgfqkeys{/tikz/arrow inside}{#1}},
	set arrow inside={end/.initial=>, opt/.initial=},
	/pgf/decoration/Mark/.style={
		mark/.expanded=at position #1 with
		{
			\noexpand\arrow[\pgfkeysvalueof{/tikz/arrow inside/opt}]{\pgfkeysvalueof{/tikz/arrow inside/end}}
		}
	},
	arrow inside/.style 2 args={
		set arrow inside={#1},
		postaction={
			decorate,decoration={
				markings,Mark/.list={#2}
			}
		}
	},
}

\usepackage{mathtools} % for \xhookrightarrow

% For fake coproduct symbol
\DeclareRobustCommand{\coprod}{\mathop{\text{\fakecoprod}}}
\newcommand{\fakecoprod}{%
	\sbox0{$\prod$}%
	\smash{\raisebox{\dimexpr.9625\depth-\dp0}{\scalebox{1}[-1]{$\prod$}}}%
	\vphantom{$\prod$}%
}

% For vertical equivalence symbol
\newcommand{\vequiv}{\rotatebox{90}{$\equiv$}}
\newcommand{\hdash}{\rotatebox{90}{$|$}}

% For quotes
\usepackage{epigraph}
\renewcommand{\epigraphsize}{\small}

\let\originalepigraph\epigraph 
\renewcommand\epigraph[2]{\originalepigraph{#1}{\textsc{#2}}}

% For tikzcd
\tikzcdset{
	/tikz/commutative diagrams/every diagram/.append style={
		row sep=huge,
		column sep=huge
	},
	row sep/subtext/.initial = 1pt,
}

\usepackage{musicography} % For music natural symbol

\DeclareMathAlphabet{\mathcal}{OMS}{cmsy}{m}{n} % Reset mathcal font to the default one
